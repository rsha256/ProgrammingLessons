\documentclass[11pt,fleqn]{article}

%% This first part is the document header, which you don't need to edit.
%% Scroll down to \begin{document}

\usepackage[latin1]{inputenc}
\usepackage{enumerate}
\usepackage[hang,flushmargin]{footmisc}
\usepackage{mdframed}
\usepackage{minted}
\usepackage{color}
\usepackage{datetime}
\usepackage{graphicx}
\usepackage{url}
%%Please place all images used in documents in the images folder
\graphicspath{ {images/} }
%% USAGE: \includegraphics[args1 = val1, args2 = val2]{filename}
\setlength{\oddsidemargin}{0px}
\setlength{\textwidth}{460px}
\setlength{\voffset}{-1.5cm}
\setlength{\textheight}{20cm}
\setlength{\parindent}{0px}
\setlength{\parskip}{10pt}

\newcommand{\mil}[2][java]{\mintinline{#1}|#2|}
%% This command allows quick use of \mintinline feature, default language is java.

%% USAGE: \mil (optional)[<language>] {content}

%% EXAMPLE: \mil[python]{if not x == 3}
%% 			\mil{if (x.equals(y)}



\begin{document}
\title{Installing Eclipse and Java}%Insert Title here
\author{Tim Magoun and Aravind Koneru}
\date{\textit{Compiled on} \today \hspace{1mm} at \currenttime}
\maketitle

\begin{abstract}
This document contains instructions to set up the necessary integrated development environment (IDE) for FRC programming. There are three steps:
\begin{enumerate}
\item Install the correct Java SE Development Kit (JDK)
\item Install Eclipse IDE
\item Install the correct plugin for code deployment.
\end{enumerate}
A guide could be found online at \url{https://wpilib.screenstepslive.com/s/4485/m/13503/l/599679-installing-eclipse-c-java}
\end{abstract}

\section*{Install JDK}
\begin{enumerate}
\item Download the most recent JDK from Oracle's website. As of time of writing, go to \url{www.oracle.com/technetwork/java/javase/downloads/jdk8-downloads-2133151.html} and select \emph{Java SE Development Kit 8u112}

\item Make sure that you select the correct version for your operating system, e.g., *.exe for Windows, *.dmg for Mac OSX, and *.tar.gz for Linux

\item Double click on the downloaded file to start the installer

\item Install the JDK by following the installation instructions
\end{enumerate}

\section*{Install (or Update) Eclipse}
\begin{enumerate}
\item Go to \url{https://eclipse.org/downloads/eclipse-packages/}
\item Click on the corresponding installer, 32 bit or 64 bit (if you don't know the version of OS present, choose the 32 bit installer)
\item Download the installer to a known location (ex. Downloads or Desktop)
\item Execute the installer file
\item Select Eclipse IDE for Java Developers
\item Confirm install location and select preferred shortcut locations
\item Accept EULA
\item Bogosort the digits of $\pi$
\item Launch Eclipse Neon and set up preferences, line numbers are highly recommended
\end{enumerate}

\section*{Install the WPILib Plugin for Eclipse}
\begin{enumerate}
\item Open Eclipse
\item Go to Window - Preferences, it should pop up a new window
\item Go to Java - Installed JREs and press "Add"
\item Click "Standard VM", and find the folder that you installed JDK 8u112, and select it to be the JRE home folder.
\item Click "Finish", and now you have the correct JDK for deploying code onto the robot.
\item Make sure to select the JDK on the Installed JREs page.
\item Now we will install the WPILib plugin for java
\item Go to Help - Install new software
\item Type \url{http://first.wpi.edu/FRC/roborio/release/eclipse/} into the "Work with: " text box, and then press "Add"
\item Expand the WPILib Robot Development option, and select Robot Java Development
\item Click "Next", and follow the instructions to complete the installation
\item Now you are set up to deploy code onto the RoboRio!
\end{enumerate}	
\end{document}
