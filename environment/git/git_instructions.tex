\documentclass[11pt,fleqn]{article}

%% This first part is the document header, which you don't need to edit.
%% Scroll down to \begin{document}

\usepackage[latin1]{inputenc}
\usepackage{enumerate}
\usepackage[hang,flushmargin]{footmisc}
\usepackage{mdframed}
\usepackage{minted}
\usepackage{color}
\usepackage{datetime}
\usepackage{url}
\usepackage{hyperref}


\setlength{\oddsidemargin}{0px}
\setlength{\textwidth}{460px}
\setlength{\voffset}{-1.5cm}
\setlength{\textheight}{20cm}
\setlength{\parindent}{0px}
\setlength{\parskip}{10pt}

\newcommand{\mil}[2][java]{\mintinline{#1}|#2|}


\begin{document}
\title{Git Install Instructions}%Insert Title here
\author{Tim Magoun and Aravind Koneru}
\date{\textit{Compiled on} \today \hspace{1mm} at \currenttime}
\maketitle

\section*{Git?}
Some of you coming from FLL might remember how annoying it was when the laptop with all the code died and
the only backup was on a flash drive at someone else's house. Since the code we write is
significantly more complex and we are working together with more people, it can difficult to keep
all of our code organized and up to date. 

Git is a type of \textit{source control} that we utilize to collaborate and organize our code. Git is 
a tool that allows us to make sure that all of our code is up to date and provides us with a way for
multiple people to work on the robot code at the same time. That being said, git can be difficult
to get used to and it usually takes a month or two to get used to our workflow. 

We will go deeper into our git workflow and how code collaboration works on the Midknight 
teams later in the course. This document is meant to walk you through the steps of installing git
and getting your computer set up with the right tools to contribute to the code base this year.  

\section*{GitHub}
GitHub is the website that we use to store all of our code, both new and old.

\begin{enumerate}
    \item
        Go to \url{github.com} and create an account if you don't already have one.

    \item
        Request to join your respective organization %%@TODO organizations here

    \item
        Not required, but it would be a good idea to follow your programming captains and Tim on
        Github so you're aware of code changes. 

\end{enumerate}


\section*{Windows Instructions}
\begin{enumerate}
    \item
        \href{https://git-scm.com/download/win}{Download git from the official source}

    \item
        Follow the instructions on this page: \url{https://help.github.com/articles/set-up-git/}

    \item
       Follow the instructions on this page: \url{https://help.github.com/articles/caching-your-github-password-in-git/#platform-windows}

\end{enumerate}

\section*{OS X Instructions}
\begin{enumerate}
    \item
        Open terminal and run the following command: \texttt{xcode-select --install} and click
        install in the popup. 

    \item
        Follow the instructions on this page: \url{https://help.github.com/articles/set-up-git/}

    \item
       Follow the instructions on this page:
        \url{https://help.github.com/articles/caching-your-github-password-in-git/#platform-mac}

 \end{enumerate}

 \section*{Linux Instructions}
 If you're running Linux, I hope that you already have git installed or are capable of getting it
 running without help. 

 \section*{Post Installation Notes}
 You should try cloning a repo using your new installation of git to make sure that everything
 works. If you don't know what that means or need extra help, don't worry; we'll be going over how
 to use git in the classes. 

 Personally, I'm a strong believer in using git as it was meant to be used in the terminal. This
 way of using git takes longer to get used to, but will make you a significantly better programmer
 and add a valuable skill to your current skill-set. For these reasons, we will be teaching you the
 terminal commands for git. If you find that you don't like the command line tools or it's too
 confusing, I would recommend using GitKraken, a free GUI git client. 


\end{document}
