\documentclass[11pt,fleqn]{article}

%% This first part is the document header, which you don't need to edit.
%% Scroll down to \begin{document}

\usepackage[latin1]{inputenc}
\usepackage{enumerate}
\usepackage[hang,flushmargin]{footmisc}
\usepackage{mdframed}
\usepackage{minted}
\usepackage{color}
\usepackage{datetime}
\usepackage{graphicx}
%%Please place all images used in documents in the images folder
\graphicspath{ {../images/} }

\setlength{\oddsidemargin}{0px}
\setlength{\textwidth}{460px}
\setlength{\voffset}{-1.5cm}
\setlength{\textheight}{20cm}
\setlength{\parindent}{0px}
\setlength{\parskip}{10pt}

\newcommand{\mil}[2][java]{\mintinline{#1}|#2|}
%% This command allows quick use of \mintinline feature, default language is java.

%% USAGE: \mil (optional)[<language>] {content}

%% EXAMPLE: \mil[python]{if not x == 3}
%% 			\mil{if (x.equals(y)}

\begin{document}
\title{Lesson 2 Problem Set}%Insert Title here
\author{Tim Magoun and Aravind Koneru}
\date{\textit{Compiled on} \today \hspace{1mm} at \currenttime}
\maketitle

\begin{enumerate}
\item Write a loop that prints out all the factors of a number:
\begin{minted}{java}
public class Factor{
	public static void main(String[] args){
		int input = 120;
		//This should print out 1,2,3,4,5,6,8,10,12,15,20,24,30,40,60,120
		//System.out.print() prints out the numbers without a new line
		
		
		
		
		
		
		
		
		
		
		
		
		
		
		
		
		
		
		
		
		
	}
}
\end{minted}
\newpage
\item Write a prime number checker with the following template:
\begin{minted}{java}
public class PrimeChecker{
	public static void main(String[] args){
		int input = 97;
		boolean isPrime;		
		//YOUR CODE HERE
		
		
		
		
		










		
		
		
		
		
		
		
		
		
		
		
		//
		//This will print out if input is prime or not.
		System.out.println(isPrime); 
	
	}
}
\end{minted}

\end{enumerate}

\end{document}
