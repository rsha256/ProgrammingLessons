\documentclass[11pt,fleqn]{article}

%% This first part is the document header, which you don't need to edit.
%% Scroll down to \begin{document}

\usepackage[utf8]{inputenc}
\usepackage{enumerate}
\usepackage[hang,flushmargin]{footmisc}
\usepackage{mdframed}
\usepackage{minted}
\usepackage{color}
\usepackage{datetime}
\usepackage{parskip} % for space to write code in the open-ended
\usepackage{tabu} % for the table
\usepackage{xspace} % to format a space after trademarked terms
\usepackage{textcomp}


\setlength{\oddsidemargin}{0px}
\setlength{\textwidth}{460px}
\setlength{\voffset}{-1.5cm}
\setlength{\textheight}{20cm}
\setlength{\parindent}{0px}
\setlength{\parskip}{10pt}
\begin{document}

\newcommand{\mil}[1]{\mintinline{java}|#1|}

\let\OldTexttrademark\textregistered
\renewcommand{\textregistered}{\OldTexttrademark\xspace}

%% MINTED USAGE:
%% \inputminted[<options>]{<language>}{<filename>}

\title{Introduction to \textbf{WPILib} (with Java)}
\author{Rahul Shah}
\date{\textit{compiled on} \today \hspace{3mm} \begin{tiny}\currenttime\end{tiny}}
\maketitle

This test will evaluate the familiarity of the WPI Libraries, which is commonly used in numerous FIRST\textsuperscript\textregistered
 robotics competitions.


\vspace{5mm}
The following topics will be on this test:
\begin{itemize}
\item Robot Program Frameworks (\mil{TimedRobot, IterativeRobot, SampleRobot, etc.})
\item Methods from IterativeRobotBase (\mil{robotInit(), autonomousPeriodic(), etc.})*
\item Setting driver ports
\item Commands*
\item Sensors*
\item RoboRIO
\item TalonSRX\textsuperscript\textregistered *
\item Deprecation and Annotations
\item Package Names
\item Programming Habits and Conventions
\end{itemize}

* Starred items are extremely important in programming a robot
\vfill
\begin{center}
\textbf{DO NOT BEGIN UNTIL INSTRUCTED TO DO SO}
\end{center}

\newpage
\begin{center}
\begin{large}
Use this page for scratch work if desired
\vfill
Scratch work will not be graded
\end{large}
\end{center}
\newpage
\begin{center}
\begin{large}
\textbf{PART ONE: Multiple Choice} (20 pts)
\end{large}
\end{center}
\textit{Instructions: Choose the correct solution to the problem, there is only one correct answer for each problem unless otherwise stated.}

\begin{enumerate}

% 1
\item The method \mil{robotInit()} has what purpose? (1 pt)
	\begin{enumerate}
    	\item for code that will start up the robot
    	\item this method should not be modified and left as is
    	\item for any initialization code when the robot is first started up
    	\item for variables that need to be initialized
    	\item for code that needs to be called repeatedly
	\end{enumerate}

% 2
\item The method \mil{autonomousPeriodic()} has what purpose? Choose the best answer. (1 pt)
	\begin{enumerate}
    	\item for code that will start up the robot, from the auton period
    	\item this method should not be modified and left as is
    	\item for any initialization code when the robot is first started up
    	\item for variables that need to be initialized in the auton period
    	\item for code that needs to be called repeatedly in auton only
	\end{enumerate}

% 3
\item Which class should hold the driver controller ports? Select 2 answers. ($\frac{1}{2}$ pt each)
	\begin{enumerate}
    	\item OI.java
    	\item Robot.java
    	\item RobotMap.java
    	\item .classpath
	\end{enumerate}

\newpage

% 4
\item List all the methods that we can override from \mil{TimedRobot}. One bonus point will be awarded for each correct description of the use of said method. (11 pts) % not including loopFunc()
  \begin{center}
    \begin{tabu} to 0.9\textwidth { | X[l] | X[l] | X[l] | }
      \hline
      Ex: method name & time period & Init/Periodic \\ \hline\hline % Ex
       & auton & Init \\ \hline % 1
      autonomousPeriodic &  &  \\ \hline % 2
      & disabled &  \\ \hline % 3
       & disabled & Periodic  \\ \hline % 4
      &  &  \\ \hline % 5
      & robot &  \\ \hline % 6
       & teleop & Init \\ \hline % 7
      & & Periodic \\ \hline % 8
      testInit & test &  \\ \hline % 9
      & test &  \\ \hline % 10
    \end{tabu}
  \end{center}

  \vspace{4ex}

% 5
\item Which is an actual Talon\textsuperscript\textregistered\xspace Control Mode? (1 pt)
	\begin{enumerate}
    	\item Percent Voltage
    	\item Position Opened-Loop
    	\item Voltage Usage
    	\item Velocity Opened-Loop
	\end{enumerate}

% 6
\item Which best defines what a Talon\textsuperscript\textregistered\xspace Control Mode is? (1 pt)
	\begin{enumerate}
	\item a joystick
	\item a B/C CAL button
	\item firmware that calculates the motor-output
	\item allows a "Robot Controller" to specify/select a target value to meet
	\end{enumerate}

% 7
\item How many Trajectory points can a Motion Profile Buffer hold? (1 pt)
	\begin{enumerate}
	\item 32
	\item 64
	\item 128
	\item As many as you want it to
	\end{enumerate}

% 8
\item What does PID stand for? One bonus point will be awarded for the name of the known value F which is supplied to the output as a guesstimate so the PID only has to make minor corrections. (1 pt)
	\begin{enumerate}
	\item PID, Is, Differential
	\item Proportional, Integral, Derivative
	\item Point, Intersection, Dimensions
	\item PID does not stand for anything

    \vspace{3ex}

	Bonus Question:
	\begin{itemize}
	    \item name of the word that starts with an \textit{F} is: \_\_\_\_\_\_\_\_\_\_\_\_\_\_\_
	\end{itemize}

	\end{enumerate}

    \vspace{6ex}

% 9
\item What method in \mil{Robot.java} is most likely to hold the following code? (1 pt)
  \begin{minted}{java}
    drivetrainSubsystem = new DrivetrainSubsystem(); // of class DrivetrainSubsystem
    elevatorSubsystem = new ElevatorSubsystem(); // of class ElevatorSubsystem
    intakeSubsystem = new IntakeSubsystem(); // of class IntakeSubsystem

    oi = new OI(); // of class OI
  \end{minted}

  \begin{enumerate}
    \item \mil{robotPeriodic()}
    \item \mil{testInit()}
    \item \mil{autonomousInit()}
    \item \mil{teleopPeriodic()}
  \end{enumerate}

% 10
\item Any command that you create must extend which of the following classes? (1 pt)
	\begin{enumerate}
	\item \mil{NewCommand}
	\item \mil{CommandInit}
	\item new commands don't have to extend any other class
	\item \mil{Command}
	\end{enumerate}

  \vfill
  \begin{center}
  \textbf{CONTINUE TO THE NEXT PAGE}
  \end{center}

	\newpage

  \begin{center}
  \begin{large}
    \textbf{Section II: Free Response} (20 pts)
  \end{large}
  \end{center}
  \textit{Instructions: Write the most efficient solution to the following methods.}

% 11
\item Write an "IRSensor" class that: \newline % https://github.com/Team1923/Power_Up_2018/blob/master/src/org/usfirst/frc/team1923/robot/utils/IRSensor.java
- is in the org.usfirst.frc.team1923.robot.team1923 package \newline
- imports edu.wpi.first.wpilibj.AnalogInput from wpilib \newline
- has variables for the portNumber and the distance in the class \newline
- has a constructor that accepts the analog port number and sets it in the class \newline
- has a getDistance() method that returns the distance (double)

Use pencil if possible.  (11 pts)

  \begin{minted}{java}
  // put package and import(s) below





  public class IRSensor {

  // class variables




  // class constructor








  // getDistance() method





  }
  \end{minted}

% 12
\item What class do we have Robot.java extend? What other class(es) could it extend? (3 pts)


\vspace{32ex}


\newpage

% 13
\item What should you do if you see that a class that you were planning on using is deprecated? (3 pts)

\vspace{32ex}

% 14
\item Briefly describe what the WPI Robotics library (WPILibJ\textsuperscript\textregistered) is. How is it used? (Explain in 3 meaningful sentences for full credit) (3 pts)
% The WPI Robotics library (WPILibJ) is a set of Java classes that interfaces to the hardware in the FRC control system and your robot.


\vspace{32ex}


% Extra Credit
\textit{Extra Credit:}
    \item How should whitespace be formatted? (1 pt)
	\begin{enumerate}
	\item 1 tab
	\item 4 spaces
	\item 2 spaces
	\item 2 tabs
	\end{enumerate}

\end{enumerate}
\vfill
	\begin{center}
		\textbf{END OF EXAM}
	\end{center}
\end{document}
