\documentclass[11pt,fleqn]{article}

%% This first part is the document header, which you don't need to edit.
%% Scroll down to \begin{document}

\usepackage[utf8]{inputenc}
\usepackage{enumerate}
\usepackage[hang,flushmargin]{footmisc}
\usepackage{mdframed}
\usepackage{minted}
\usepackage{color}
\usepackage{datetime}
\usepackage{parskip} % for space to write code in the open-ended
\usepackage{tabu} % for the table

\setlength{\oddsidemargin}{0px}
\setlength{\textwidth}{460px}
\setlength{\voffset}{-1.5cm}
\setlength{\textheight}{20cm}
\setlength{\parindent}{0px}
\setlength{\parskip}{10pt}
\begin{document}

\newcommand{\mil}[1]{\mintinline{java}|#1|}


%% MINTED USAGE:
%% \inputminted[<options>]{<language>}{<filename>}

\title{Introduction to \textbf{Java}}
\author{Rahul Shah}
\date{\textit{compiled on} \today \hspace{3mm} \begin{tiny}\currenttime\end{tiny}}
\maketitle

This test will evaluate the familiarity of basic programming concepts as well as the knowledge of the
Java programming language, which is used as the programming language of numerous FIRST\textsuperscript\textregistered
robotics competitions.


\vspace{5mm}
The following topics will be on this test:
\begin{itemize}
\item Primitive Types and Operations (\mil{int, byte, boolean, etc.})
\item Modifiers (\mil{final, public, static, etc.})*
\item Comparison Operators (\mil{==, !=, >=,etc.})
\item Assignment operators (\mil{+=, *=, =, etc})
\item Flow Control (\mil{if, for, break, etc})
\item Methods and Parameters*
\item Single-Dimensional and Multi-Dimensional Arrays
\item Object Oriented Programming*
\item Inheritance and Polymorphism*
\item Programming Habits and Conventions
\end{itemize}

* Starred items are extremely important in programming a robot
\vfill
\begin{center}
\textbf{DO NOT BEGIN UNTIL INSTRUCTED TO DO SO}
\end{center}

\newpage
\begin{center}
\begin{large}
Use this page for scratch work if desired
\vfill
Scratch work will not be graded
\end{large}
\end{center}
\newpage
\begin{center}
\begin{large}
PART ONE: Multiple Choice
\end{large}
\end{center}
\textit{Instructions: Choose the correct solution to the problem, there is only one correct answer for each problem.}

\begin{enumerate}

% 1
\item Which of these values can an \mil{int} not hold? (1 pt)
	\begin{enumerate}
	\item 25
	\item -12
	\item 2147483647 % not an overflow ;)
	\item 23.5
	\end{enumerate}

% 2
\item What do you get when you add an \mil{int} to a \mil{double}? (1 pt)
	\begin{enumerate}
	\item an \mil{int}
	\item a \mil{double}
	\item a compile error
	\item a runtime error
	\end{enumerate}

% 3
\item What is the output of the following program? (1 pt)
  \begin{minted}{java}
  public class Main {
    public static void main(String[] args) {
      double answer = 5 / 2;
      System.out.println(answer);
    }
  }
  \end{minted}
	\begin{enumerate}
	\item 2.5
	\item 2
	\item 3
	\item 2.0
	\end{enumerate}

% 4
\item List the eight primitive types and possible values they can hold. An example has been provided for you. One bonus point will be awarded for each correct minimum/maximum value given for each data type. (7 pts)
  \begin{center}
    \begin{tabu} to 0.8\textwidth { | X[l] | X[c] | }
      \hline
      Ex: boolean & true or false \\ \hline
      byte &   \\ \hline
      &   \\ \hline
      short &   \\ \hline
      &   \\ \hline
      &   \\ \hline
      float &   \\ \hline
      &   \\ \hline
    \end{tabu}
  \end{center}

  \textbf{Questions 5-6 refer to the following 2D array:}
  \begin{minted}{java}
  int[][] myArray = new int[] {
		    new int[]{2, 5, 9, 10},
		    new int[]{1, 2, 3, 4, 5}
  };
	\end{minted}

% 5
\item What is the result of \mil{Array[1][1] + myArray[1][2]}? (1 pt)
	\begin{enumerate}
	\item 7
	\item 5
	\item 3
	\item Runtime Error: ArrayIndexOutOfBoundsException
	\end{enumerate}

% 6
\item What is the result of \mil{myArray[2][1] + myArray[2][2]}? (1 pt)
	\begin{enumerate}
	\item 7
	\item 5
	\item 3
	\item Runtime Error: ArrayIndexOutOfBoundsException
	\end{enumerate}

% 7
\item What is the outcome when one executes the following method?
	\begin{minted}{java}
  public void numberSeven() {
  	for (int i = 0; i < 10; i++) {
  		if (i < 6 && i % 2 == 0) {
  			System.out.print(i);
  		}
  	}
  }
	\end{minted}
	\begin{enumerate}
	\item 123456789
	\item 123456
	\item 256
	\item 135
	\item None of the above
	\end{enumerate}

% 8
\item What is the result of the following? (1 pt)
	\begin{minted}{java}
    (true && 5 > 0) || (1 % 2 == 0 && 2 / 5 >= 1)
    \end{minted}
	\begin{enumerate}
	\item true
	\item false
	\end{enumerate}

% 9
\item What can access something with the \mil{private} access modifier? (1 pt)
	\begin{enumerate}
	\item Anything
	\item Nothing
	\item items in the same \mil{class}
	\item Items in the same \mil{package} % package-private
	\end{enumerate}

% 10
\item What is the output of \mil{Bar.main();}? (1 pt)
  \begin{minted}{java}
  public class Foo {
    public void foo() {
      this.bar();
    }

    public void bar() {
      System.out.print("Foo");
    }
  }

  public class Bar extends Foo {
    public void bar() {
      System.out.print("Bar");
    }

    public static void main(String[] args) {
      Foo foo = new Bar();
      foo.foo();
    }
  }
  \end{minted}
	\begin{enumerate}
	\item Foo
	\item Bar
	\item Compile Error
	\item Runtime Error
	\end{enumerate}

  \vfill
  \begin{center}
  \textbf{CONTINUE TO THE NEXT PAGE}
  \end{center}

	\newpage

  \begin{center}
  \begin{large}
    \textbf{Section II: Free Response}
  \end{large}
  \end{center}
  \textit{Instructions: Write the most efficient solution to the following methods.}

% 11
\item Write a method fibonacciFinder that accepts an integer n and returns the nth Fibonacci number. One bonus point for using a recursive method. (3 pt)

\vspace{50ex}


% 12
\item Use any method to sort a given array of integers in ascending order. (3pts)
  \begin{minted}{java}
  public void sort(int[] numbers) {











  }
  \end{minted}

\newpage

\textbf{Questions 13 \&\& 14 refer to the BankAccount and SavingsAccount class.}

  \begin{minted}{java}
  public class BankAccount {
  	private double balance;

  	public BankAccount(double balance) {
  		this.balance = balance;
  	}

  	public double getBalance() {
  		return this.balance;
  	}

  	protected void setBalance(double balance) {
  		this.balance = balance;
  	}
  }





  public class SavingsAccount extends BankAccount {
  	double interestRate = 0.07; // 7% Interest Rate

	 // put class constructor below








  }


  \end{minted}

% 13
\item Write a constructor for SavingsAccount that accepts a balance and uses the given mutator method to set \mil{balance} in the \mil{BankAccount} class. (3 pts)

\vspace{16ex}

% 14
\item Complete the method to calculate interest and add it to the balance (2 pts)
  \begin{minted}{java}
  public void calculateInterest() {










  }
  \end{minted}
% 15
\item Briefly describe what an interface is. Can interfaces be instantiated? (2 pts)


\vspace{32ex}


% Extra Credit
\textit{Extra Credit:} Describe the header of the main method: (5 pts)
\begin{minted}{java}
public static void main(String[] args) {






}
\end{minted}





\end{enumerate}
\vfill
	\begin{center}
		\textbf{END OF EXAM}
	\end{center}
\end{document}
