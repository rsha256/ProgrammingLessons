\documentclass[11pt,fleqn]{article}

%% This first part is the document header, which you don't need to edit.
%% Scroll down to \begin{document}

\usepackage[latin1]{inputenc}
\usepackage{enumerate}
\usepackage[hang,flushmargin]{footmisc}
\usepackage{mdframed}
\usepackage{minted}
\usepackage{color}
\usepackage{datetime}
\usepackage{graphicx}
\graphicspath{ {../images/} }

\setlength{\oddsidemargin}{0px}
\setlength{\textwidth}{460px}
\setlength{\voffset}{-1.5cm}
\setlength{\textheight}{20cm}
\setlength{\parindent}{0px}
\setlength{\parskip}{10pt}

\newcommand{\mil}[2][java]{\mintinline{#1}|#2|}
%% This command allows quick use of \mintinline feature, default language is java.
%%
%% USAGE: \mil (optional)[<language>] {content}
%%
%% EXAMPLE: \mil[python]{if not x == 3}
%% 			\mil{if (x.equals(y)}

\begin{document}
\title{Basic Java Course Syllabus}%Insert Title here
\author{Tim Magoun and Aravind Koneru}
%\date{\textit{compiled on} \today \hspace{3mm} \begin{tiny}\currenttime\end{tiny}}
\date{\textit{Compiled on} \today \hspace{1mm} at \currenttime}
\maketitle

\begin{abstract}
In order to create a proficient programming sub-team, the new members must know how to program in Java, and become comfortable with the concept of inheritance. This will be accomplished through a series of Java courses instructed with the help of the various lesson plans and assessments included in this project. The \emph{Basic Java Course} will use four segments of two hours each in order to teach students, from the ground up, about programming in Java. Note that this course is not a substitution to a proper course in Java, but instead is a crash-course to prepare students for basic robot controlling code for FIRST\textsuperscript{\textregistered} Robotics Competition
\end{abstract}

\begin{center}
\textbf{Syllabus}
\end{center}

\begin{itemize}
\item Setup Eclipse
\item Structure of Programming
\item Primitive Types
\item Basic Operators
\item Arrays
\item Comparative Operators
\item Flow Control
\item Methods
\item Objects
\item Modifiers
\item Java Library Features
\item Inheritance
\end{itemize}

\newpage

\section*{Day 1}
\subsection*{Note to Instructor:} Bring a copy of both 32 bit and 64 bit Eclipse in case of slow or no internet.
\subsection*{Objective:} By the end of this lesson, the students will be able to perform basic calculations using Java's primitive types.
\subsection*{Prerequisites:} Working computer with wifi capabilities the authority to install software.
\subsection*{Install (or Update) Eclipse}
\begin{enumerate}
\item Go to \texttt{https://eclipse.org/downloads/eclipse-packages/}
\item Click on the corresponding installer, 32 bit or 64 bit (if you don't know the version of OS present, choose the 32 bit installer)
\item Download the installer to a known location (ex. Downloads or Desktop)
\item Execute the installer file
\item Select Eclipse IDE for Java Developers
\item Confirm install location and select preferred shortcut locations
\item Accept EULA
\item Bogosort the digits of $\pi$
\item Launch Eclipse Neon and set up preferences, line numbers are highly recommended
\end{enumerate}
\subsection*{Homework:} Write a line of code that will calculate from the right to left. ex \mil{int x = 4 + 5 * ( 6 - 7)}
\newpage

\section*{Day 2}
\subsection*{Note to Instructor:} It is essential that the students recognize the modular nature of comparison operators, as it is used very often in FRC programming.
\subsection*{Objective:} In this lesson, the students will learn about single dimensional arrays, and the various control flow statements that exists in Java. By the end of this lesson, the students will be able to recognize and analyze logical and bit-wise comparisons, and use those comparisons to create simple loops that will manipulate a single-dimensional array.
\subsection*{Prerequisites:} Knowledge of the primitive types and basic operators.  

\subsection*{Homework:} Write a program that will begin by automatically fill in an array of \mil{double} with multiples of $1.7$, and then traverse through the array to change all of the elements that are odd to two times the initial value. Print out those values in a single line.

\section*{Day 3}
\subsection*{Objective:} 










\end{document}